\documentclass[a4paper,11pt]{article}
\usepackage[top=0.5cm, bottom=1.5cm, left=1.3cm, right=1.3cm]{geometry}
\usepackage[colorlinks = true, citecolor=red, urlcolor=blue]{hyperref}

\parindent=2pt
\parskip=1ex plus 0.5ex minus 0.2ex

\begin{document}

\title{Repeated games with mistakes}
\author{Nikolas M. Skoufis \\ Supervisor: Julian Garcia}
\date{October 20th, 2015}

\maketitle

\section*{Abstract}

This project studies efficient methods to compute the expected payoff of an iterated prisoner's dilemma between two opponents.
In order to efficiently compute these payoffs, the mathematics underpinning the iterated prisoners dilemma and the computation of expected payoff was analyzed with a view to improving in speed of computation over existing methods.
A Monte-Carlo method of computing expected payoff was used as a benchmark against which two fast methods for computing payoff were compared.
The results of these fast methods were then used to generate three-population simplex plots of evolutionary dynamics as a further benchmark.

\section*{Introduction}

The iterated prisoner's dilemma is an archetypal problem in game theory, particularly in the study of the evolution of cooperation.
The iterated prisoner's dilemma involves two prisoners, each of which has two moves available to them.
The prisoner's are told of the two options available to them, and they are told that they have 24 hours to make a decision or they will be punished.
Each player can either inform on the other player (known as \textit{defection}) or stay silent (known as \textit{cooperation}).
If both prisoners defect, will both spend 10 years in prison.
If both prisoners cooperate, they will each spend only 5 years in prison; a better outcome for both.
However if one prisoner cooperates and the other defects, the prisoner who defected goes free, and the prisoner that cooperated will spend 20 years in prison; longer than any other option.

Given these options and outcomes, we can see that a natural dilemma arises.
If a player cooperates, they may receive a better outcome than if they had defected, depending on the other player's action.
However if they defect, they have the opportunity for best possible outcome (going free), but they also risk punishment if the other player defects.
This problem is the prisoner's dilemma.

An important concept in game theory is that of the Nash equilibrium.
The Nash equilibrium is a set of strategies where each player cannot improve their own outcome by changing their strategy while the opponent's strategy remains the same.
This means that in effect, a Nash equilibrium is the optimal result for all players.
It is a well known result in game theory that the only Nash equilibrium in the prisoner's dilemma is for both players to defect.
However this result is only valid for the single-shot prisoner's dilemma ie\. the prisoner's dilemma where the players only play the game once.
If players player multiple times, more complex strategies can emerge and there emerges an incentive for mutual cooperation~\cite{trivers}.

\begin{thebibliography}{9}

    \bibitem{garciaandtraulsen}
        Julian Garcia, Arne Traulsen,
        \emph{The Structure of Mutations and the Evolution of Cooperation},
        PLoS ONE,
        Vol. 7,
        No. 4,
        2012.

    \bibitem{trivers}
        Robert Trivers,
        \emph{The Evolution of Reciprocal Altruism},
        The Quarterly Review of Biology,
        Vol. 46,
        No. 1,
        1971.

    \bibitem{hypothesis}
        David R. MacIver,
        \emph{Hypothesis},
        https://github.com/DRMacIver/hypothesis,
        2015.

    \bibitem{ross}
        Sheldon M. Ross,
        \emph{Simulation},
        Fourth edition,
        Elsevier Academic Press,
        2006.

\end{thebibliography}

\end{document}
